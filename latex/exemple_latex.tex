\documentclass{book}

\usepackage[utf8]{inputenc}	% Encodage
\usepackage[T1]{fontenc}		% Encodage
\usepackage[francais]{babel}	% Langue
\usepackage{lmodern}		% Police
\usepackage{url}			% Urls cliquables

\usepackage{fancyhdr}		% Entête & pied de page personnalisés

% Pour afficher les marges : \usepackage{layout} 
\usepackage[top=5cm, bottom=5cm, left=6cm, right=3cm]{geometry}		%Modifier les marges

\title{Les droits de l'homme}		% Page de garde
\author{\textsc{Thibaud Colas}}		% Page de garde
\date{\today}					% Page de garde

\begin{document}

\pagestyle{fancy}

\renewcommand{\headheight}{13pt}  
\renewcommand{\headrulewidth}{2pt} 
\renewcommand{\footrulewidth}{2pt} 
\renewcommand{\headsep}{15pt}
\lhead{\rightmark }
\chead{}
\rhead{\leftmark }
\lfoot{Section \thesection}
\cfoot{Chapitre \thechapter }
\rfoot{\thepage}

\maketitle					% Page de garde
%Marges : \layout

\frontmatter				% Préface / Préambule
\chapter{Introduction}
\begin{it}
	\large{
Les droits de l'homme sont un concept selon lequel tout être humain possède des droits universels, inaliénables, quel que soit le droit positif en vigueur ou les autres facteurs locaux tels que l'ethnie, la nationalité, l'orientation sexuelle ou la religion.

Selon cette philosophie, combattue ou éclipsée aux 19\up{ième} siècle, 20\up{ième} siècle et 21\up{ième} siècle par d’autres doctrines, l’homme, en tant que tel, et indépendamment de sa condition sociale, a des droits \og inhérents à sa personne, inaliénables et sacrés \fg, et donc opposables en toutes circonstances à la société et au pouvoir. Ainsi le concept de droits de l’homme est-il par définition universaliste et égalitaire, incompatible avec les systèmes et les régimes fondés sur la supériorité ou la \og vocation historique \fg d’une caste, d’une race, d’un peuple, d’une classe ou d’un quelconque groupe social ; incompatible tout autant avec l’idée que la construction d’une société meilleure justifie l’élimination ou l’oppression de ceux qui sont censés faire obstacle à cette édification.
Les droits de l'homme, types de prérogatives dont sont titulaires les individus, sont généralement reconnus dans les pays occidentaux par la loi, par des normes de valeur constitutionnelle ou par des conventions internationales, afin que leur respect soit assuré, si besoin est même contre l'État. L'existence, la validité et le contenu des droits de l'homme sont un sujet permanent de débat en philosophie et en sciences politiques.
	}
\end{it}

\mainmatter				% Corps

\part{Déclaration des droits de l’homme et du citoyen de 1789}

\chapter{Histoire}
La Déclaration des Droits de l'Homme a été écrite dans un temple protestant. L’assemblée réunie à Versailles par la convocation des États généraux pour trouver une solution fiscale au déficit de l’État, se déclare Assemblée nationale en réunissant les trois ordres, dont elle décide l’abolition, puis s’institue Assemblée nationale constituante, et décide de rédiger une déclaration des principes fondamentaux à partir desquels sera établie une nouvelle Constitution. Elle se réunit pour cela, après avoir pris les décrets des 4 et 11 août 1789 sur la suppression des droits féodaux, qu’elle reprendra dans l’article premier de la Déclaration.

La Déclaration des droits de l’Homme et du Citoyen a été débattue par l’Assemblée nationale française sous la présidence du marquis de \textsc{Mirabeau} à partir d’un des trois projets proposés, celui de 24 articles rédigé par le VI\up{e} bureau, dirigé par Jérôme \textsc{Champion de Cicé}. L’attribution du texte primitif à \textsc{La Fayette} inspiré par la Déclaration d'indépendance des États-Unis est donc erronée. L’abbé Grégoire propose que la Déclaration des droits de l’Homme et du Citoyen soit accompagnée de celle des devoirs.

La discussion débute le 9 juillet et débouche sur un vote le 26 août 1789, sous l’influence des leaders du tiers-état et de la noblesse. Le texte est peu modifié, mais est enrichi d’un préambule. Il est ratifié en partie le soir du 5 octobre 1789 par Louis XVI à Versailles, sur l’exigence de l’Assemblée, qui utilisa la pression d’une foule vindicative venue de Paris, initialement pour d’autres revendications.
Entièrement promulguée par le Roi à Paris, le 3 novembre 1789, la Déclaration des Droits est la dernière ordonnance royale. Elle servira de Préambule à la première Constitution de la Révolution française.

\chapter{Le texte}
\section{Introduction}

Les représentants du peuple français, constitués en Assemblée nationale, considérant que l’ignorance, l’oubli ou le mépris des droits de l’homme sont les seules causes des malheurs publics et de la corruption des gouvernements, ont résolu d’exposer, dans une déclaration solennelle, les droits naturels, inaliénables et sacrés de l’homme, afin que cette déclaration, constamment présente à tous les membres du corps social, leur rappelle sans cesse leurs droits et leurs devoirs ; afin que les actes du pouvoir législatif et ceux du pouvoir exécutif, pouvant être à chaque instant comparés avec le but de toute institution politique, en soient plus respectés ; afin que les réclamations des citoyens, fondées désormais sur des principes simples et incontestables, tournent toujours au maintien de la Constitution et au bonheur de tous.

\section{Les articles}
\paragraph{Article premier}
Les hommes naissent et demeurent libres et égaux en droits. Les distinctions sociales ne peuvent être fondées que sur l'utilité commune.
\paragraph{Article 2}
Le but de toute association politique est la conservation des droits naturels et imprescriptibles de l'homme. Ces droits sont la liberté, la propriété, la sûreté et la résistance à l'oppression.
\paragraph{Article 3}
Le principe de toute souveraineté réside essentiellement dans la Nation. Nul corps, nul individu ne peut exercer d'autorité qui n'en émane expressément.
\paragraph{Article 4}
La liberté consiste à pouvoir faire tout ce qui ne nuit pas à autrui : ainsi, l'exercice des droits naturels de chaque homme n'a de bornes que celles qui assurent aux autres membres de la société la jouissance de ces mêmes droits. Ces bornes ne peuvent être déterminées que par la loi.
\paragraph{Article 5}
La loi n'a le droit de défendre que les actions nuisibles à la société. Tout ce qui n'est pas défendu par la loi ne peut être empêché, et nul ne peut être contraint à faire ce qu'elle n'ordonne pas.
\paragraph{Article 6}
La loi est l'expression de la volonté générale. Tous les citoyens ont droit de concourir personnellement ou par leurs représentants à sa formation. Elle doit être la même pour tous, soit qu'elle protège, soit qu'elle punisse. Tous les citoyens, étant égaux à ces yeux, sont également admissibles à toutes dignités, places et emplois publics, selon leur capacité et sans autre distinction que celle de leurs vertus et de leurs talents.
\paragraph{Article 7}
Nul homme ne peut être accusé, arrêté ou détenu que dans les cas déterminés par la loi et selon les formes qu'elle a prescrites. Ceux qui sollicitent, expédient, exécutent ou font exécuter des ordres arbitraires doivent être punis ; mais tout citoyen appelé ou saisi en vertu de la loi doit obéir à l'instant ; il se rend coupable par la résistance.
\paragraph{Article 8}
La loi ne doit établir que des peines strictement et évidemment nécessaires, et nul ne peut être puni qu'en vertu d'une loi établie et promulguée antérieurement au délit, et légalement appliquée.
\paragraph{Article 9}
Tout homme étant présumé innocent jusqu'à ce qu'il ait été déclaré coupable, s'il est jugé indispensable de l'arrêter, toute rigueur qui ne serait pas nécessaire pour s'assurer de sa personne doit être sévèrement réprimée par la loi.
\paragraph{Article 10}
Nul ne doit être inquiété pour ses opinions, mêmes religieuses, pourvu que leur manifestation ne trouble pas l'ordre public établi par la loi.
\paragraph{Article 11}
La libre communication des pensées et des opinions est un des droits les plus précieux de l'homme ; tout citoyen peut donc parler, écrire, imprimer librement, sauf à répondre de l'abus de cette liberté dans les cas déterminés par la loi.
\paragraph{Article 12}
La garantie des droits de l'homme et du citoyen nécessite une force publique ; cette force est donc instituée pour l'avantage de tous, et non pour l'utilité particulière de ceux à qui elle est confiée.
\paragraph{Article 13}
Pour l'entretien de la force publique, et pour les dépenses d'administration, une contribution commune est indispensable ; elle doit être également répartie entre les citoyens, en raison de leurs facultés.
\paragraph{Article 14}
Les citoyens ont le droit de constater, par eux-mêmes ou par leurs représentants, la nécessité de la contribution publique, de la consentir librement, d'en suivre l'emploi, et d'en déterminer la quotité, l'assiette, le recouvrement et la durée.
\paragraph{Article 15}
La société a le droit de demander compte à tout agent public de son administration.
\paragraph{Article 16}
Toute société dans laquelle la garantie des droits n'est pas assurée ni la séparation des pouvoirs déterminée, n'a point de Constitution.
\paragraph{Article 17}
La propriété étant un droit inviolable et sacré, nul ne peut en être privé, si ce n'est lorsque la nécessité publique, légalement constatée, l'exige évidemment, et sous la condition d'une juste et préalable indemnité.


\chapter{Sources}

La question des sources de la Déclaration française a suscité une controverse empreinte de nationalisme au sein de l’historiographie. Dans une brochure de 1895, l’historien allemand Georg \textsc{Jellinek} présentait l’\oe uvre française comme une simple héritière des Déclarations anglo-saxonnes (\emph{Pétition des droits}, \emph{Déclaration des droits}), elles-mêmes inspirées du Protestantisme luthérien. Traduite en français en 1902, dans un contexte de montée des tensions entre France et Allemagne, elle donnera lieu à une réplique aussi peu nuancée, portée par Emile \textsc{Boutmy} : la Déclaration des droits de l’homme et du citoyen n’aurait de source que dans la tradition philosophique et humaniste des Lumières.

Le Préambule, ajouté au projet, a été rédigé sous l’influence de \textsc{Mirabeau}, et de Jean-Joseph \textsc{Mounier}, député du Tiers qui avait fait adopter le serment du Jeu de Paume, tous deux monarchiens, c’est-à-dire partisans d’une Monarchie constitutionnelle à l’anglaise.
L’invocation à l’ \og Être suprême \fg a été rajoutée au cours des séances pour tenir compte des convictions chrétiennes de presque tous les citoyens.

Le texte de l’article Un, \og Tous les Hommes naissent et demeurent libres et égaux en droit \fg, synthétise la Loi du 4 août 1789 abolissant la société d’ordres hiérarchisés.

L’article 16, associant constitution et organisation de la séparation des pouvoirs, est un principe antérieurement admis avec la séparation des ordres spirituel, politique et économique. Mais les trois pouvoirs politiques auxquels renvoie implicitement cet article, à savoir le législatif, l’exécutif et le judiciaire, est la conception proposée par \textsc{Montesquieu} depuis 1748 dans \emph{De l’Esprit des Lois}.

L’article 3, qui attribue la souveraineté à la Nation, s’inspire des thèmes des remontrances des Parlements, portées par les nombreux membres du club des Amis de la Constitution, plus connu sous le nom de Club des Jacobins, mais aussi du célèbre pamphlet de l’abbé \textsc{Sieyès}, qui pose l’équation : peuple = Tiers-État, c’est-à-dire que les députés du Tiers-État sont les représentants légitimes de l’ensemble du peuple.

L’article 6, directement inspiré du philosophe \textsc{Rousseau}, a été proposé par \textsc{Talleyrand}. Lu à la tribune du comité de constitution le 12 septembre 1789, ce qui deviendra l’article 6 de la déclaration des droits prenait la forme suivante : \og La loi étant l’expression de la volonté générale, tous les citoyens ont droit de concourir personnellement ou par représentation à sa formation ; elle doit être la même pour tous \fg.

Les autres articles reprennent des principes généraux du droit ou de la procédure qui sont déjà établis, comme la positivité du droit, le caractère contradictoire des procédures, la non rétroactivité des lois, etc.

Son idéal est l’individualisme libéral. C’est une \oe uvre de circonstance, une proclamation générale, un texte tourné vers le passé avec pour objectif d’en finir avec l’Ancien Régime ; mais également un texte tourné vers l’avenir en promouvant la philosophie des lumières et son idéal rationaliste.

\appendix					% Appendice
\chapter{La révolution française}

La Révolution française est la période de l’histoire de France comprise entre la convocation des États généraux en 1789 et le coup d’État du 18 brumaire (9-10 novembre 1799) de Napoléon \textsc{Bonaparte}. C’est un moment fondamental de l’histoire de France, marquant la fin de l’Ancien Régime, et le passage à une monarchie constitutionnelle puis à la Première République. Elle a mis fin à la royauté, à la société d’ordres et aux privilèges. Justifié par la Déclaration des droits de l’homme et du citoyen, qui proclamait l’égalité théorique des citoyens devant la loi, les libertés fondamentales et la souveraineté de la Nation, apte à se gouverner au travers des représentants élus, cette période causa la mort de plusieurs milliers de personnes et la terreur pour la majorité.

\backmatter				% Postface / Annexe
 
\chapter{Sources et licences}
    
\section*{Sources}

\begin{description}
	\item[Droits de l'Homme (Wikipedia) :] \url{http://bit.ly/9nYoSU}
	\item[Déclaration de 1789 (Wikipedia) :] \url{http://bit.ly/1DJqQg}
	\item[Déclaration de 1793 (Wikipedia) :] \url{http://bit.ly/SeRyG}
	\item[Déclaration de 1795 (Wikipedia) :] \url{http://bit.ly/b6sdRI}
	\item[La révolution française (Wikipedia) :] \url{http://bit.ly/cUd3SF}
\end{description}
    
\section*{Licence Creative Commons 3.0}

\paragraph{Vous êtes libres :}    	
\begin{itemize}
	\item de reproduire, distribuer et communiquer cette création au public
	\item de modifier cette création
\end{itemize}
  	
\paragraph{Selon les conditions suivantes :}
\begin{itemize}
	\item \emph{Paternité} : vous devez citer le nom de l'auteur original de la manière indiquée par l'auteur de l'\oe uvre ou le titulaire des droits qui vous confère cette autorisation (mais pas d'une manière qui suggérerait qu'ils vous soutiennent ou approuvent votre utilisation de l'\oe uvre).
	\item \emph{Partage des conditions initiales à l'identique} : si vous transformez ou modifiez cette \oe uvre pour en créer une nouvelle, vous devez la distribuer selon les termes du même contrat ou avec une licence similaire ou compatible.
\end{itemize}


\end{document}